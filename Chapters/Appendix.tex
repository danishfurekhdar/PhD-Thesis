\chapter{Derivations in Dipole Approximation}\label{appendix-A}

This appendix presents the detailed mathematical derivations of two fundamental components in strong-field quantum dynamics: the Volkov phase and the transition amplitude, both treated within the dipole approximation. The dipole approximation, which neglects the spatial dependence of the laser field ($\mathbf{A}(\mathbf{r},t) \approx \mathbf{A}(t)$), is valid when the electron's quiver amplitude is small compared to the laser wavelength.

\section{Plane Wave Pulse}
It is convenient to represent the electromagnetic field, particularly when working with time-varying fields or in situations where the electric and magnetic fields are orthogonal, in terms of a vector potential. For a sine-squared pulse envelope, the vector potential in the dipole approximation is given by:
\begin{equation}
    \begin{aligned}
        \mathbf{A}(t) = \frac{A_{0}}{\sqrt{1+\epsilon^{2}}} f(t) \biggl(
        &\cos(\omega t + \phi_{\text{CEP}}) \mathbf{e}_{x} \\
        + \epsilon\Lambda &\sin(\omega t + \phi_{\text{CEP}}) \mathbf{e}_{y} \biggr),
    \end{aligned}
    \label{Eq:A1}
\end{equation}
where $\epsilon$ represents the ellipticity, $\Lambda$ is the helicity ($\Lambda = \pm 1$), $\omega$ is the carrier frequency, $\phi_{\text{CEP}}$ is the carrier-envelope phase (CEP), and $A_0$ is the peak amplitude.

The envelope function $f(t)$, which describes the temporal shape of the pulse, is given by:
\begin{equation}
    f(t) = \begin{cases}
        \sin^{2}\left(\frac{\omega t}{2n_{p}}\right), & 0 \leq t \leq \tau_{p} \\
        0,                                           & \text{otherwise},
    \end{cases}
    \label{Eq:A2}
\end{equation}
where $n_p$ is the number of optical cycles within the pulse duration $\tau_p = n_p T$ ($T = 2\pi/\omega$ being the optical period).

By expanding the trigonometric products and substituting \eqref{Eq:A2} into \eqref{Eq:A1}, the vector potential can be expressed as:
\begin{equation}
    \mathbf{A}(t) = \frac{A_{0}}{\sqrt{1+\epsilon^{2}}} 
    \left[
        \begin{aligned}
            -\frac{1}{4} 
            &\left(
                \begin{aligned}
                    &\cos\left(\omega t \left(1-\frac{1}{n_p}\right) + \phi_{\text{CEP}}\right) \\
                    &+ \cos\left(\omega t \left(1+\frac{1}{n_p}\right)+ \phi_{\text{CEP}}\right) \\
                    &- 2\cos\left(\omega t + \phi_{\text{CEP}}\right)
                \end{aligned}
            \right)\mathbf{e}_{x} \\[2ex]
            -\frac{\epsilon\Lambda}{4} 
            &\left(
                \begin{aligned}
                    &\sin\left(\omega t \left(1-\frac{1}{n_p}\right) + \phi_{\text{CEP}}\right) \\
                    &+ \sin\left(\omega t \left(1+\frac{1}{n_p}\right) + \phi_{\text{CEP}}\right) \\
                    &- 2\sin\left(\omega t + \phi_{\text{CEP}}\right)
                \end{aligned}
            \right)\mathbf{e}_{y}
        \end{aligned}
    \right],
\end{equation}

In a more compact form, the vector potential can be written as a sum over three frequency components:
\begin{equation}
    \mathbf{A}(t) = \sum_{j=0}^{2} \frac{\mathcal{A}_j}{\sqrt{1+\epsilon^2}} 
    \left[ \cos(\omega_j t + \phi_{\text{CEP}}) \mathbf{e}_x 
    + \epsilon\Lambda \sin(\omega_j t + \phi_{\text{CEP}}) \mathbf{e}_y \right],
    \label{eq:VectorPotentialCompactAppendix}
\end{equation}
where we have introduced the following notation for the frequencies and amplitudes:
\begin{equation}
		\omega_{j} = \left(1-\frac{1}{n_p}, \quad 1, \quad 1+\frac{1}{n_p}\right)\omega, \quad \text{and} \quad \mathcal{A}_j = \left(-\frac{1}{4}, \quad \frac{1}{2}, \quad -\frac{1}{4}\right)A_0.
\end{equation}

\section{Dipole Volkov Phase for elliptically polarized plane wave pulse}\label{sec:volkovphaseinellip}
The Volkov phase in the dipole approximation describes the quantum mechanical phase acquired by a free electron in an electromagnetic field. It is given by the time integral of the kinetic energy:
\begin{equation}
S(\mathbf{p},t) = \frac{1}{2} \int^{t} (\mathbf{p} + \mathbf{A}(\tau))^2 \; d\tau.
\label{eq:VolkovPhaseGeneral}
\end{equation}
where $\mathbf{p}$ is the canonical momentum and $\mathbf{A}(\tau)$ is the vector potential of the laser field.
The integral in Eq.~\eqref{eq:VolkovPhaseGeneral} can be decomposed into three distinct terms:
\begin{equation}
    \begin{aligned}
        I   &= \frac{1}{2} \int^{t} \mathbf{p}^2 \, d\tau, \\
        II  &= \int^{t} \mathbf{p} \cdot \mathbf{A}(\tau) \, d\tau, \\
        III &= \frac{1}{2} \int^{t} \mathbf{A}(\tau) \cdot \mathbf{A}(\tau) \, d\tau.
    \end{aligned}
    \label{eq:VolkovComponents}
\end{equation}
The first term represents the free-electron kinetic energy contribution
\begin{equation}
    I = \frac{\mathbf{p}^2}{2} t,
    \label{eq:TermI}
\end{equation}
which is straightforward to evaluate.
Using the vector potential from Eq.~\eqref{eq:VectorPotentialCompactAppendix}, we evaluate the second term
\begin{align}
II &=  \sum_{j=0}^{2} \frac{\mathcal{A}_j}{\sqrt{1+\epsilon^2}} \int^{t}
    \left[ \cos(\omega_j \tau + \phi_{\text{CEP}}) \mathbf{p} \cdot \mathbf{e}_x 
    + \epsilon\Lambda \sin(\omega_j \tau + \phi_{\text{CEP}}) \mathbf{p} \cdot \mathbf{e}_y \right] \; d\tau \nonumber\\
  &= \sum_{j=0}^{2} \frac{\mathcal{A}_j}{\sqrt{1+\epsilon^2}} 
    \left[ p_x \int^{t} \cos(\omega_j \tau + \phi_{\text{CEP}}) \; d\tau 
    + \epsilon\Lambda p_y \int^{t} \sin(\omega_j \tau + \phi_{\text{CEP}}) \; d\tau \right] \; d\tau \nonumber\\
& =  \sum_{j=0}^{2} \frac{\mathcal{A}_j}{\sqrt{1+\epsilon^2}} 
    \left[ \frac{p_x}{\omega_j} \sin(\omega_j \tau + \phi_{\text{CEP}})
    - \epsilon\Lambda \frac{p_y}{\omega_j} \cos(\omega_j \tau + \phi_{\text{CEP}})  \right].
\end{align}
Now we can derive the third integral
\begin{align}
III = \frac{1}{2} \int^{t} \mathbf{A}_i(\tau) \cdot \mathbf{A}_j(\tau) \; d\tau.
\end{align}
The third term represents the ponderomotive energy and requires more careful analysis. We separate it into two cases
\paragraph{Case A: Diagonal Terms ($i=j$)}
\begin{align}
III-A &= \frac{1}{2} \int^{t} \mathbf{A}_j^2(\tau) \; d\tau. \nonumber \\
	  & = \frac{1}{2} \int^{t} \frac{\mathcal{A}^2_j}{1+\epsilon^2}\int^{t} [\cos^2(\omega_j \tau + \phi_{\text{CEP}}) + \epsilon^2 \sin^2(\omega_j \tau + \phi_{\text{CEP}})] \; d\tau \nonumber\\
	  & = \frac{1}{2} \int^{t} \frac{\mathcal{A}^2_j}{1+\epsilon^2}\int^{t} [\frac{1}{2}\left(1 + \cos(2\omega_j \tau + 2\phi_{\text{CEP}})\right) +  \frac{\epsilon^2}{2}\left(1-\cos(2\omega_j \tau + 2\phi_{\text{CEP}})\right)] \; d\tau  \nonumber \\
	  &= \frac{\mathcal{A}^2_j}{4}\int^{t} d\tau  + \frac{\mathcal{A}^2_j}{4} \frac{1-\epsilon^2}{1 + \epsilon^2}\int^{t}\cos(2\omega_j \tau + 2\phi_{\text{CEP}})\; d\tau \nonumber \\
	  &= \frac{\mathcal{A}^2_j}{4} t + \frac{\mathcal{A}^2_j}{8\omega_j} \frac{1-\epsilon^2}{1 + \epsilon^2} \sin(2\omega_j t + 2\phi_{\text{CEP}}),
\end{align}  
here, we used the identity $\cos^2\theta = \frac{1}{2}(1 + \cos2\theta)$ and $\sin^2\theta = \frac{1}{2}(1 - \cos2\theta)$.
\paragraph{Case B: Off-Diagonal Terms ($i \neq j$)}
\begin{align}
III-B = \sum_{i=0}^{2} \sum_{j=0,j \neq i}^{2} \frac{\mathcal{A}_i\mathcal{A}_j}{2(1+\epsilon^2)} \int^{t} &[\cos(\omega_i \tau + \phi_{\text{CEP}})\cos(\omega_j \tau + \phi_{\text{CEP}}) \nonumber \\
 &+ \epsilon^2 \sin(\omega_i \tau + \phi_{\text{CEP}})\sin(\omega_j \tau + \phi_{\text{CEP}})] \; d\tau.
\end{align}
Now to further simplify we can use the following trigonometric identities
\begin{align}
&\cos(a)\cos(b) = \frac{1}{2}(cos(a-b) + cos(a+b)) \quad \text{and} \nonumber \\ &\sin(a)\sin(b) = \frac{1}{2}(cos(a-b) - cos(a+b)). \nonumber
\end{align} 
To simplify our integral we will again divide it into two parts $III-B^1$ and $III-B^2$. Starting with the first part
\begin{align}
III-B^1 &= \sum_{i=0}^{2} \sum_{j=0,j \neq i}^{2} \frac{\mathcal{A}_i\mathcal{A}_j}{2(1+\epsilon^2)} \int^{t}\frac{1}{2}[\cos((\omega_i - \omega_j)\tau) + \cos((\omega_i + \omega_j)\tau + 2\phi_{\text{CEP}})]d\tau \nonumber \\
		&= \sum_{i=0}^{2} \sum_{j=0,j \neq i}^{2} \frac{\mathcal{A}_i\mathcal{A}_j}{4(1+\epsilon^2)} \left[\frac{\sin((\omega_i - \omega_j)t)}{\omega_i - \omega_j} + \frac{\sin((\omega_i + \omega_j)t + 2\phi_{\text{CEP}})}{\omega_i + \omega_j}\right]
		\label{eq:part1}
\end{align}
similarly the second part will become
\begin{align}
III-B^2 &= \sum_{i=0}^{2} \sum_{j=0,j \neq i}^{2} \frac{\mathcal{A}_i\mathcal{A}_j}{2(1+\epsilon^2)} \int^{t}\frac{\epsilon^2}{2}[\cos((\omega_i - \omega_j)\tau) - \cos((\omega_i + \omega_j)\tau + 2\phi_{\text{CEP}})]d\tau \nonumber \\
		&= \sum_{i=0}^{2} \sum_{j=0,j \neq i}^{2} \frac{\mathcal{A}_i\mathcal{A}_j}{4(1+\epsilon^2)} \epsilon^2\left[\frac{\sin((\omega_i - \omega_j)t)}{\omega_i - \omega_j} - \frac{\sin((\omega_i + \omega_j)t + 2\phi_{\text{CEP}})}{\omega_i + \omega_j}\right]
\label{eq:part2}
\end{align}
Noting that the double sum over $i \neq j$ can be written more efficiently as $2\sum_{i=0}^1 \sum_{j=i+1}^2$, we combine these results
\begin{align}
III-B = &\sum_{i=0}^{1} \sum_{j=i+1}^{2}\frac{\mathcal{A}_i\mathcal{A}_j}{2(\omega_i - \omega_j)} \sin((\omega_i - \omega_j)t) \nonumber \\
&\frac{1-\epsilon^2}{1+\epsilon^2}\sum_{i=0}^{1} \sum_{j=i+1}^{2}\frac{\mathcal{A}_i\mathcal{A}_j}{2(\omega_i + \omega_j)} \sin((\omega_i + \omega_j)t + 2\phi_{\text{CEP}})
\label{eq:part3B}
\end{align}
Combining all terms, we arrive at the complete expression for the Volkov phase in the dipole approximation
\begin{eqnarray}
    S(\mathbf{p}, t) &=& \frac{1}{2} \mathbf{p}^2 t
    + \frac{t}{4} \sum_{i=0}^{2} \mathcal{A}_i^2 
    + \frac{1 - \epsilon^2}{1 + \epsilon^2} \sum_{i=0}^{2} \frac{\mathcal{A}_i^2}{8\omega_i} \sin(2\omega_i t + 2\phi_{\mathrm{CEP}}) \nonumber \\
    && + \sum_{i=0}^{1} \sum_{j=i+1}^{2} \frac{\mathcal{A}_i \mathcal{A}_j}{2 (\omega_i - \omega_j)} \sin((\omega_i - \omega_j)t) \nonumber \\
    && + \frac{1 - \epsilon^2}{1 + \epsilon^2} \sum_{i=0}^{1} \sum_{j=i+1}^{2} \frac{\mathcal{A}_i \mathcal{A}_j}{2 (\omega_i + \omega_j)} \sin((\omega_i + \omega_j)t + 2\phi_{\mathrm{CEP}}) \nonumber \\
    && + \frac{p_x}{\sqrt{1 + \epsilon^2}} \sum_{i=0}^{2} \frac{\mathcal{A}_i}{\omega_i} \sin(\omega_i t + \phi_{\mathrm{CEP}}) \nonumber \\
    && - \epsilon \Lambda \frac{p_y}{\sqrt{1 + \epsilon^2}} \sum_{i=0}^{2} \frac{\mathcal{A}_i}{\omega_i} \cos(\omega_i t + \phi_{\mathrm{CEP}}).
    \label{Eq:VolkovPhaseSolAppendix}
\end{eqnarray}
This expression captures all essential physics of the electron dynamics in the laser field, including the free electron motion, ponderomotive energy contributions, and field-dressing effects.


\section{Transition Amplitude in Dipole Approximation}
For a hydrogen-like \( 1s \) wave function, the initial bound state can be effectively approximated using a modified ionization potential \( I_p \). This allows us to express the time-dependent wave function as
\begin{equation}
	|\Psi_0 (t) \rangle = |\Psi_0 \rangle e^{i I_p t} = \frac{(2I_p)^{3/2}}{\sqrt{\pi}} e^{-\sqrt{2 I_p} r} e^{i I_p t}.
	\label{Eq:initialWavefunctionappendix}
\end{equation}
This expression represents the bound state, where the wave function maintains an exponential decay in space and an oscillatory phase factor in time due to the ionization potential \( I_p \).

Together with Eq. \ref{eq:VolkovPhaseGeneral} and Eq. \ref{Eq:initialWavefunctionappendix} we can write the direct transition amplitude in velocity gauge as
\begin{equation}
	T_0(\mathbf{p}) = - \langle \mathbf{p} | \Psi_0 \rangle e^{-\dot\iota (S(\mathbf{p}, t)+ I_p t)} \Big|_{0}^{\tau_p}  - \dot\iota \langle \mathbf{p} | V(\mathbf{r}) | \Psi_0 \rangle\int_{0}^{\tau_p} dt \; e^{-\dot\iota (S(\mathbf{p}, t)+ I_p t)}.
	\label{Eq:DirectTransitionAmplitudeappendix}
\end{equation}
The matrix element of the Coulomb potential and the initial wave function in momentum space can be expressed as 
\begin{equation}
	\langle \mathbf{p} | V(\mathbf{r}) | \Psi_0 \rangle = - \frac{2^{\frac{3}{2}} I_p^{\frac{5}{4}}}{\pi} \frac{1}{\epsilon_p + I_p}.
\end{equation}
Similarly, the initial wave function in momentum space is given by:
\begin{equation}
	\langle \mathbf{p} | \Psi_0 \rangle = \frac{(2 I_p)^{\frac{5}{4}}}{\pi \sqrt{2}} \frac{1}{\left( \epsilon_p + I_p \right)^2}.
\end{equation}
Now the remaining task is to evaluate the time integral in the second part of equation \ref{Eq:DirectTransitionAmplitudeappendix}. Such a integral is not easy to solve because of the phase factor given in the equation \ref{Eq:VolkovPhaseSolAppendix}. The exponential \( e^{-iS(\mathbf{p}, t)} \) can be expanded using the Jacobi-Anger expansion, yielding:
\begin{align}
    e^{-iS(\mathbf{p}, t)} &= \exp\left(-i \left( \frac{1}{2} \mathbf{p}^2 t + \frac{t}{4} \sum_{i=0}^{2} \mathcal{A}_i^2 \right) \right) \nonumber \\
    &\times \prod_{i=0}^{2} \sum_{n_i = -\infty}^{\infty}  J_{n_i}\left( \frac{1 - \epsilon^2}{1 + \epsilon^2} \frac{\mathcal{A}_i^2}{8\omega_i} \right) e^{-i n_i (2\omega_i t + 2\phi_{\mathrm{CEP}})} \nonumber \\
    &\times \prod_{i=0}^{1} \prod_{j=i+1}^{2} \sum_{m_{ij} = -\infty}^{\infty}  J_{m_{ij}}\left( \frac{\mathcal{A}_i \mathcal{A}_j}{2 (\omega_i - \omega_j)} \right) e^{-i m_{ij} (\omega_i - \omega_j) t} \nonumber \\
    &\times \prod_{i=0}^{1} \prod_{j=i+1}^{2} \sum_{k_{ij} = -\infty}^{\infty}  J_{k_{ij}}\left( \frac{1 - \epsilon^2}{1 + \epsilon^2} \frac{\mathcal{A}_i \mathcal{A}_j}{2 (\omega_i + \omega_j)} \right) e^{-i k_{ij} ((\omega_i + \omega_j) t + 2\phi_{\mathrm{CEP}})} \nonumber \\
    &\times \prod_{i=0}^{2} \sum_{\ell_i = -\infty}^{\infty} J_{\ell_i}\left( \frac{p_x \mathcal{A}_i}{\omega_i \sqrt{1 + \epsilon^2}} \right) e^{-i \ell_i (\omega_i t + \phi_{\mathrm{CEP}})} \nonumber \\
    &\times \prod_{i=0}^{2} \sum_{q_i = -\infty}^{\infty} (-i)^{q_i} J_{q_i}\left( \frac{\epsilon \Lambda p_y \mathcal{A}_i}{\omega_i \sqrt{1 + \epsilon^2}} \right) e^{i q_i (\omega_i t + \phi_{\mathrm{CEP}})},
\end{align}
where \( J_n(\cdot) \) are Bessel functions of the first kind, and the expansion has been applied to each oscillatory term in \( S(\mathbf{p}, t) \). Now the integral is more straight forward as we got rid of the trigonometric functions in terms of Bessel function which are time independent. The integral of the exponents are now straight forward using the general way that $\int e^{kx}dx = \frac{e^{kx}}{k}$. Thus applying the expansion and general defination of integrals we can derive the final transition amplitude as
\begin{align}
	T^D_0(\mathbf{p}) = \prod_{i=1}^{15} \left\{ (-\dot\iota)^{n_i \Theta(i-12)} \sum_{n_i=-\infty}^{\infty} J_{n_i} (x_i^D) \right. 
	&&  \left. \left( \frac{\langle \mathbf{p} | V(\mathbf{r}) | \Psi_0 \rangle}{\epsilon_p + U_p + \Omega_{n_i}^D + I_p} + \langle \mathbf{p} | \Psi_0 \rangle \right) \right. \nonumber\\ 
	&& \left. \left( 1 - e^{\dot\iota \left[\left(\epsilon_p + U_p + \Omega_{n_i}^D +I_p\right) \tau_p + \Phi_i^D\right]} \right) \right\},
	\label{Eq:TransitionAmplitudeJacobiappendix}
\end{align}
where \( \Theta(x) \) is the Heaviside step function which ensures that the extra factor \( i^{n_i} \) appears only for the last three terms (\( i \geq 13 \)), while it remains absent for the first 12 terms (\( i \leq 12 \)).The D in the superscript represents the Dipole case and the arguments of the Bessel function in the transition amplitude \ref{Eq:TransitionAmplitudeJacobiappendix} include terms involving the squared vector potential amplitudes, their respective frequencies, and interaction terms between different frequency components, adjusted by the ellipticity factor. These terms are explicitly given as
\begin{eqnarray}
	x^D &=& \left( \frac{1 - \epsilon^2}{1 + \epsilon^2}\frac{\mathcal{A}_0^2}{8\omega_0}, \frac{1 - \epsilon^2}{1 + \epsilon^2}\frac{\mathcal{A}_1^2}{8\omega_1}, \frac{1 - \epsilon^2}{1 + \epsilon^2}\frac{\mathcal{A}_2^2}{8\omega_2},  \frac{\mathcal{A}_0 \mathcal{A}_1}{2(\omega_0 - \omega_1)}, \frac{\mathcal{A}_0 \mathcal{A}_2}{2(\omega_0 - \omega_2)}, \frac{\mathcal{A}_1 \mathcal{A}_2}{2(\omega_1 - \omega_2)}, \right.  \nonumber \\
	&& \left. \frac{1 - \epsilon^2}{1 + \epsilon^2}\frac{\mathcal{A}_0 \mathcal{A}_1}{2(\omega_0 + \omega_1)}, \frac{1 - \epsilon^2}{1 + \epsilon^2}\frac{\mathcal{A}_0 \mathcal{A}_2}{2(\omega_0 + \omega_2)}, \frac{1 - \epsilon^2}{1 + \epsilon^2}\frac{\mathcal{A}_1 \mathcal{A}_2}{2(\omega_1 + \omega_2)},\quad \right.  \nonumber \\
	&& \left. \frac{p_x}{\sqrt{1 + \epsilon^2}}\frac{\mathcal{A}_0}{\omega_0}, \frac{p_x}{\sqrt{1 + \epsilon^2}}\frac{\mathcal{A}_1}{\omega_1}, \frac{p_x}{\sqrt{1 + \epsilon^2}}\frac{\mathcal{A}_2}{\omega_2},\right.  \nonumber \\
	&& \left. \epsilon \Lambda \frac{p_y}{\sqrt{1 + \epsilon^2}}\frac{\mathcal{A}_0}{\omega_0}, \epsilon \Lambda \frac{p_y}{\sqrt{1 + \epsilon^2}}\frac{\mathcal{A}_1}{\omega_1}, \epsilon \Lambda \frac{p_y}{\sqrt{1 + \epsilon^2}}\frac{\mathcal{A}_2}{\omega_2}, \right),
\end{eqnarray}
where the first set of terms represents the contributions from individual frequency components, while the remaining terms describe cross-interactions between them.
The modified frequency and phase, essential for describing the transition dynamics, are defined as
\begin{eqnarray}
	\Omega^D &=& \left(
	2n_1\omega_0 + 2n_2\omega_1 + 2n_3\omega_2 + n_4(\omega_0 - \omega_1) + n_5(\omega_0 - \omega_2) + n_6(\omega_1 - \omega_2) \right.\nonumber \\
	& & \left. + n_7(\omega_0 + \omega_1) + n_8(\omega_0 + \omega_2) + n_9(\omega_1 + \omega_2) + n_{10}\omega_0 + n_{11}\omega_1 + n_{12}\omega_2 \right.\nonumber \\
	& & \left. + n_{13}\omega_0 + n_{14}\omega_1 + n_{15}\omega_2 \right) \\
	\Phi^D &=& \left(
	2n_1\phi_{\mathrm{cep}} + 2n_2\phi_{\mathrm{cep}} + 2n_3\phi_{\mathrm{cep}} + 2n_7\phi_{\mathrm{cep}} + 2n_8\phi_{\mathrm{cep}} + 2n_9\phi_{\mathrm{cep}} + n_{10}\phi_{\mathrm{cep}} \right.\nonumber \\
	& & \left. + n_{11}\phi_{\mathrm{cep}} + n_{12}\phi_{\mathrm{cep}} + n_{13}\phi_{\mathrm{cep}} + n_{14}\phi_{\mathrm{cep}} + n_{15}\phi_{\mathrm{cep}} \right).
\end{eqnarray}
These expressions account for the summation and combination of different frequency components and their respective phase contributions.

\section{Dipole Volkov Phase for a Bessel Pulse}\label{Appendix:A4}
The Bessel beam carries orbital angular momentum (OAM) and thus has a non-vanishing $z$-component. The Bessel pulse can be generalized by multiplying the envelope function (as we did before in the plane wave case) to the Bessel beam as shown previously in equation \ref{eq:BesselBeam}. The resulting vector potential can be expanded similarly by expanding the trigonometric functions. The remaining task is to evaluate the phase integral, which is identical to our previous calculation in Section \ref{sec:volkovphaseinellip}. The final solution is given by
\begin{equation}
    S(\mathbf{\tilde{p}}, \tau) = \xi t + \sum_{j=1}^{9} \gamma_j \cos\left(\phi_j^{(c)} - \omega_j^{(c)} t\right) 
    + \sum_{l=1}^{13} \sigma_l \sin\left(\phi_l^{(s)} - \omega_l^{(s)} t\right).
    \label{Eq:VolkovPhaseSolutionAppendix}
\end{equation}

The coefficients are defined as:
\begin{align}
    \xi &= \frac{p^2}{2} + \frac{3}{32} A_0^2 (2\alpha_{-1}^2 + 2\alpha_{+1}^2 + \alpha_{0}^2), \\
    \gamma &= \frac{A_0}{2\omega}\left(-p_y\alpha_{+1}, p_z\alpha_{0}, p_y\alpha_{-1}, 
        \frac{n_p p_y \alpha_{+1}}{2(n_p -1)}, -\frac{n_p p_z \alpha_{0}}{2(n_p -1)}, \right. \nonumber \\
        &\quad \left. -\frac{n_p p_y \alpha_{-1}}{2(n_p -1)}, \frac{n_p p_y \alpha_{+1}}{2(n_p + 1)}, 
        -\frac{n_p p_z \alpha_{0}}{2(n_p + 1)}, -\frac{n_p p_y \alpha_{-1}}{2(n_p + 1)}\right), \\
    \sigma &= \frac{A_0}{2 \omega} \left( -\frac{A_0 n_p}{4} \left( 2 \alpha_{+1}^2 + 2 \alpha_{-1}^2 + \alpha_{0}^2 \right), 
        \frac{A_0 n_p}{32} \left( 2 \alpha_{+1}^2 + 2 \alpha_{-1}^2 + \alpha_{0}^2 \right), \right. \nonumber \\
        &\quad \left. -p_x \alpha_{+1}, \frac{3 A_0}{32} \left( \alpha_{0}^2 - 4 \alpha_{-1} \alpha_{+1} \right), 
        -p_x \alpha_{-1}, \frac{A_0 n_p}{8 (2 n_p - 1)} \left( 4 \alpha_{-1} \alpha_{+1} - \alpha_{0}^2 \right), \right. \nonumber \\
        &\quad \left. \frac{n_p p_x \alpha_{+1}}{2 (n_p - 1)}, \frac{A_0 n_p}{64 (n_p - 1)} \left( \alpha_{0}^2 - 4 \alpha_{-1} \alpha_{+1} \right),
        \frac{n_p p_x \alpha_{-1}}{2 (n_p - 1)}, \frac{A_0 n_p}{64 (n_p + 1)} \left( \alpha_{0}^2 - 4 \alpha_{-1} \alpha_{+1} \right), \right. \nonumber \\
        &\quad \left. \frac{n_p p_x \alpha_{+1}}{2(n_p + 1)}, \frac{n_p p_x \alpha_{-1}}{2(n_p + 1)}, 
        \frac{A_0 n_p}{8 (2n_p + 1)} \left(4 \alpha_{-1} \alpha_{+1} - \alpha_{0}^2\right) \right).
\end{align}

The phases and frequencies are defined as:
\begin{align}
    \omega^{c} &= \omega\left(1, 1, 1, \frac{n_p - 1}{n_p}, \frac{n_p - 1}{n_p}, \frac{n_p - 1}{n_p}, 
        \frac{n_p + 1}{n_p}, \frac{n_p + 1}{n_p}, \frac{n_p + 1}{n_p}\right), \\
    \omega^{s} &= \omega\left(-\frac{1}{n_p}, -\frac{2}{n_p}, 1, 2, 1, \frac{2n_p - 1}{n_p}, \frac{n_p - 1}{n_p}, 
        2\frac{n_p - 1}{n_p}, \frac{n_p - 1}{n_p}, \right. \nonumber \\
        &\quad \left. 2\frac{n_p + 1}{n_p}, \frac{n_p + 1}{n_p}, \frac{n_p + 1}{n_p}, \frac{2n_p + 1}{n_p}\right), \\
    \phi^{c} &= \varphi_b \left(m_\gamma - 1, m_\gamma, m_\gamma + 1, m_\gamma - 1, m_\gamma, m_\gamma + 1, 
        m_\gamma - 1, m_\gamma, m_\gamma + 1\right), \\
    \phi^{s} &= \varphi_b \left(0, 0, m_\gamma-1, 2m_\gamma, m_\gamma + 1, 2m_\gamma, m_\gamma - 1, 
        2m_\gamma, m_\gamma + 1, 2m_\gamma, m_\gamma - 1, m_\gamma + 1, 2m_\gamma\right).
\end{align}

The coefficients in the above equations are given by:
\begin{align}
    \alpha_{-1} &= \sqrt{\frac{\varkappa}{4\pi}} c_{-1} J_{m_{\gamma} + 1}(\varkappa r), \\
    \alpha_{0} &= \sqrt{\frac{\varkappa}{2\pi}} c_{0} J_{m_{\gamma}}(\varkappa r), \\
    \alpha_{+1} &= \sqrt{\frac{\varkappa}{4\pi}} c_{+1} J_{m_{\gamma} - 1}(\varkappa r), \\
    \beta_{-1} &= (m_{\gamma} - 1)\varphi_r + k_z z, \\
    \beta_{0} &= m_{\gamma}\varphi_r + k_z z, \\
    \beta_{+1} &= (m_{\gamma} + 1)\varphi_r + k_z z.
\end{align}

In the above coefficients, $c_0 = \frac{\Lambda}{\sqrt{2}} \sin \theta_k$ and $c_{\pm 1} = \frac{1}{2} (1 \pm \Lambda \cos \theta_k)$. Here, $\Lambda$ represents the corresponding helicity, and $\varkappa = k\sin\theta_k$ with $k = \omega/c$ (where $c$ is the speed of light). 

For small opening angles $\theta_k$, one may refer to the orbital angular momentum $m_l = m_\gamma - \Lambda$ of the Bessel beam. The longitudinal wave number is given by $k_z = k\cos\theta_k$, and $r$ in Cartesian coordinates is $r = \sqrt{x^2 + y^2}$. The parameter $n_p$ represents the optical cycles of the pulse, and $\omega$ is the corresponding frequency.


\chapter{Derivations in Nondipole Approximation}\label{appendix-B}
The time evolution of an electron under the influence of an external electromagnetic field is governed by the time-dependent Schrödinger equation (TDSE). When expressed in the velocity gauge formulation, the system's Hamiltonian incorporates minimal coupling between the electron and the field:
\begin{equation}
\hat{H}_{\text{le}} = \frac{1}{2} \left( \mathbf{p} + \mathbf{A}(\mathbf{r}, t) \right)^{\!2} = \frac{1}{2} \mathbf{p}^2 + \mathbf{p} \cdot \mathbf{A}(\mathbf{r}, t) + \frac{1}{2} \mathbf{A}^2(\mathbf{r}, t).
\end{equation}
In this expression, the first-order $\mathbf{A}$ term describes the electron-field coupling, while the second-order term corresponds to photon self-interaction. Note that in our real-space representation, the position operator $\hat{\mathbf{r}}$ simplifies to the coordinate $\mathbf{r}$.
The dynamics of an electron interacting with a general vector potential in the velocity gauge are described by the minimal-coupling Hamiltonian:
\begin{equation}
i \partial_t \ket{\chi_{\mathbf{p}}^{ND}(t)} = \left[ \frac{1}{2} \left( \mathbf{p} + \mathbf{A}(\mathbf{r}, t) \right)^2 \right] \ket{\chi_{\mathbf{p}}^{ND}(t)}.
\end{equation}
This equation admits approximate analytical solutions that include first-order relativistic corrections, known as nondipole Volkov states
\begin{equation}
\braket{\mathbf{r} | \chi_{\mathbf{p}}^{ND}(t)} = \frac{1}{(2\pi)^{3/2}} \, e^{i\mathbf{p} \cdot \mathbf{r}} \, e^{-i \epsilon_p t + i\Gamma(\mathbf{r}, t)}, \quad \text{with} \quad \epsilon_p = \frac{\mathbf{p}^2}{2},
\end{equation}
where the spatially dependent phase $\Gamma(\mathbf{r}, t)$ encodes nondipole corrections arising from the full spatial dependence of the vector potential.
The complete nondipole Volkov phase is defined as
\begin{align}
\Gamma(\mathbf{r}, t) &= \sum_{i=1}^{5} \Gamma_i(\mathbf{r}, t), \label{eq:B1a} \\
\Gamma_1(\mathbf{r}, t) &= \int \mathrm{d}^3\!\mathbf{k}\, \rho(\mathbf{k}) \sin\big(u(\mathbf{k}) + \theta(\mathbf{k})\big), \label{eq:B1b} \\
\Gamma_2(\mathbf{r}, t) &= \int \mathrm{d}^3\!\mathbf{k}\, \mathrm{d}^3\!\mathbf{k}'\, \Big[
\alpha^{+}(\mathbf{k}, \mathbf{k}') \sin\big(u(\mathbf{k}) + u(\mathbf{k}') + \theta^{+}(\mathbf{k}, \mathbf{k}')\big) \notag \\
&\quad + \alpha^{-}(\mathbf{k}, \mathbf{k}') \sin\big(u(\mathbf{k}) - u(\mathbf{k}') + \theta^{-}(\mathbf{k}, \mathbf{k}')\big)
\Big], \label{eq:B1c} \\
\Gamma_3(\mathbf{r}, t) &= \sum_{\pm} \frac{1}{2} \int \mathrm{d}^3\!\mathbf{k}\, \mathrm{d}^3\!\mathbf{k}'\, \rho(\mathbf{k}) \rho(\mathbf{k}') \notag \\
&\quad \times \frac{\sin\big(u(\mathbf{k}) \pm u(\mathbf{k}') + \theta(\mathbf{k}) \pm \zeta(\mathbf{k}, \mathbf{k}')\big)}
{\eta(\mathbf{k}) \pm \eta(\mathbf{k}')}, \label{eq:B1d} \\
\Gamma_4(\mathbf{r}, t) &= \sum_{\pm} \int \mathrm{d}^3\!\mathbf{k}\, \mathrm{d}^3\!\mathbf{k}'\, \mathrm{d}^3\!\mathbf{k}''\, 
\sigma(\mathbf{k}, \mathbf{k}') \alpha^{+}(\mathbf{k}', \mathbf{k}'') \notag \\
&\quad \times \frac{\sin\big(u(\mathbf{k}) \pm u(\mathbf{k}') + u(\mathbf{k}'') + \theta^{+}(\mathbf{k}', \mathbf{k}'') \pm \xi(\mathbf{k}, \mathbf{k}'')\big)}
{\eta(\mathbf{k}) \pm \eta(\mathbf{k}') + \eta(\mathbf{k}'')}, \label{eq:B1e} \\
\Gamma_5(\mathbf{r}, t) &= \sum_{\pm} \int \mathrm{d}^3\!\mathbf{k}\, \mathrm{d}^3\!\mathbf{k}'\, \mathrm{d}^3\!\mathbf{k}''\, 
\sigma(\mathbf{k}, \mathbf{k}') \alpha^{-}(\mathbf{k}', \mathbf{k}'') \notag \\
&\quad \times \frac{\sin\big(u(\mathbf{k}) \pm u(\mathbf{k}') - u(\mathbf{k}'') + \theta^{-}(\mathbf{k}', \mathbf{k}'') \pm \xi(\mathbf{k}, \mathbf{k}'')\big)}
{\eta(\mathbf{k}) \pm \eta(\mathbf{k}') - \eta(\mathbf{k}'')}. \label{eq:B1f}
\end{align}
where Eqs.~\eqref{eq:B1b} and \eqref{eq:B1c} are associated with the particle-field and field-field contributions of the nondipole Volkov phase, respectively. Moreover, Eqs.~\eqref{eq:B1d}--\eqref{eq:B1f} are interconnections of the geometric contribution with the respective particle-field and field-field contributions.

The nondipole phase $\Gamma(\mathbf{r}, t)$ consists of several key contributions:
\begin{subequations}
\begin{align}
\mathbf{p} \cdot \mathbf{A}(\mathbf{k}, t) &= \lambda(\mathbf{k}) \cos(u(\mathbf{k}) + \theta(\mathbf{k})), \\
\mathbf{a}(\mathbf{k}) \cdot \mathbf{a}(\mathbf{k}') / 4 &= \Delta^{+}(\mathbf{k}, \mathbf{k}') \, e^{i\theta^{+}(\mathbf{k}, \mathbf{k}')}, \\
\mathbf{a}(\mathbf{k}) \cdot \mathbf{a}^*(\mathbf{k}') / 4 &= \Delta^{-}(\mathbf{k}, \mathbf{k}') \, e^{i\theta^{-}(\mathbf{k}, \mathbf{k}')}, \\
-\mathbf{k} \cdot \mathbf{A}(\mathbf{k}', t) &= \sigma(\mathbf{k}, \mathbf{k}') \cos(u(\mathbf{k}') + \xi(\mathbf{k}, \mathbf{k}')), \\
\eta(\mathbf{k}) &= \mathbf{p} \cdot \mathbf{k} - \omega(\mathbf{k}), \\
\rho(\mathbf{k}) &= \lambda(\mathbf{k}) / \eta(\mathbf{k}), \\
\alpha^{\pm}(\mathbf{k}, \mathbf{k}') &= \Delta^{\pm}(\mathbf{k}, \mathbf{k}') / \left( \eta(\mathbf{k}) \pm \eta(\mathbf{k}') \right).
\end{align}
\label{eq:volkovcontributions}
\end{subequations}
For a rigorous derivation and comprehensive analysis, we refer to Refs.~\cite{Rosenberg1993} and \cite{Birger2020Thesis}  (Chapter~4).
For the vector potential given in Eq.~\eqref{eq:VectorPotentialCompactAppendix}, the system of equations in Eq.~\eqref{eq:volkovcontributions} can be solved for each individual mode \( j \), yielding the following solutions:
\begin{equation}
    \begin{aligned}
        \lambda(\mathbf{k}) &= \frac{A_{j}(\mathbf{k})}{\sqrt{1+\epsilon^{2}}} \sqrt{\mathbf{p}_{x}^{2} + \epsilon^{2}\mathbf{p}_{y}^{2}} \, \delta(\mathbf{k}-\mathbf{k}_{0}), \\
        \theta(\mathbf{k}) &= \phi_{\text{cep}} + \Lambda \arctan(\epsilon \tan \varphi_{p}), \\
        \rho(\mathbf{k}) &= \frac{A_{j}(\mathbf{k})}{\sqrt{1+\epsilon^{2}}} \frac{\sqrt{\mathbf{p}_{x}^{2} + \epsilon^{2}\mathbf{p}_{y}^{2}}}{\eta_{j}(\mathbf{k})} \delta(\mathbf{k}-\mathbf{k}_{0}), \\
        \Delta^{+}(\mathbf{k},\mathbf{k}') &= \frac{A_{j}(\mathbf{k}) A_{j}(\mathbf{k}')}{4} \frac{1 - \epsilon^{2}}{1 + \epsilon^{2}} \delta(\mathbf{k}-\mathbf{k}_{0}) \delta(\mathbf{k}'-\mathbf{k}_{0}), \\
        \Delta^{-}(\mathbf{k},\mathbf{k}') &= \frac{A_{j}(\mathbf{k}) A_{j}(\mathbf{k}')}{4} \delta(\mathbf{k}-\mathbf{k}_{0}) \delta(\mathbf{k}'-\mathbf{k}_{0}), \\
        \theta^{+}(\mathbf{k}) &= 2\phi_{\text{cep}}, \quad \theta^{-}(\mathbf{k}) = 0, \\
        \sigma(\mathbf{k},\mathbf{k}') &= 0, \quad \xi(\mathbf{k},\mathbf{k}') = \phi_{\text{cep}}, \\
        \alpha^{\pm}(\mathbf{k},\mathbf{k}') &= \frac{A_{j}(\mathbf{k}) A_{j}(\mathbf{k}')}{4} \frac{1 \mp \epsilon^{2}}{1 + \epsilon^{2}} \frac{1}{\eta_{j}(\mathbf{k}) \pm \eta_{j}(\mathbf{k}')} \delta(\mathbf{k}-\mathbf{k}_{0}) \delta(\mathbf{k}'-\mathbf{k}_{0}),
    \end{aligned}
    \label{Eq:B4}
\end{equation}
where \(\eta(\mathbf{k}) = \mathbf{p} \cdot \mathbf{k} - \omega_{k}\) and \(\omega_{k} = k c\).

To compute the phase \(\Gamma(\mathbf{r}, t)\), we substitute Eq.~\eqref{Eq:B4} into Eq.~\eqref{eq:B1a}. This leads to the explicit expression
\begin{align}
    \Gamma(\mathbf{r}, t) 
    ={}& \int \mathrm{d}^{3}\!\mathbf{k} \, 
    \frac{A_{j}(\mathbf{k})}{\sqrt{1+\epsilon^{2}}} 
    \frac{\sqrt{\mathbf{p}_{x}^{2} + \epsilon^{2} \mathbf{p}_{y}^{2}}}{\eta_{j}(\mathbf{k})} 
    \sin \left( u_{j} + \phi_{\mathrm{cep}} 
    + \Lambda \arctan(\epsilon \tan \varphi_{p}) \right) 
    \delta(\mathbf{k} - \mathbf{k}_{0}) \nonumber \\
    & + \int \mathrm{d}^{3}\!\mathbf{k} \int \mathrm{d}^{3}\!\mathbf{k}' \, 
    \frac{A_{j}(\mathbf{k}) A_{j}(\mathbf{k}')}{4} 
    \frac{1 - \epsilon^{2}}{1 + \epsilon^{2}} 
    \frac{1}{\eta_{j}(\mathbf{k}) + \eta_{j}(\mathbf{k}')} 
    \sin(u_{j} + u_{j}' + 2\phi_{\mathrm{cep}}) \nonumber \\
    & \hspace{6cm} \times \delta(\mathbf{k} - \mathbf{k}_{0}) \delta(\mathbf{k}' - \mathbf{k}_{0}) \nonumber \\
    & + \int \mathrm{d}^{3}\!\mathbf{k} \int \mathrm{d}^{3}\!\mathbf{k}' \, 
    \frac{A_{j}(\mathbf{k}) A_{j}(\mathbf{k}')}{4} 
    \frac{1}{\eta_{j}(\mathbf{k}) - \eta_{j}(\mathbf{k}')} 
    \sin(u_{j} + u_{j}') \nonumber \\
	& \hspace{6cm} \times \delta(\mathbf{k} - \mathbf{k}_{0}) \delta(\mathbf{k}' - \mathbf{k}_{0}).
    \label{eq:B5}
\end{align}
The delta functions can be used to simplify these integrals, leading to Eq.~\eqref{Eq:VolkovPhaseSolNondipole}. Note that the third integral vanishes because the wavevectors \(\mathbf{k}\) are parallel to the laser propagation direction (z-axis), making \(\sigma(\mathbf{k},\mathbf{k}') = 0\). In the above expression, one should note that we combined the terms depending on $p_x$ and $p_y$, as in the equation \ref{Eq:VolkovPhaseSolNondipole}, which give the factor $\arctan(\epsilon \tan\varphi_{p})$ where $\varphi_{p} = \frac{p_y}{p_x}$.


\chapter{Modified Saddle Point Method for Ionization Amplitudes}\label{appendix-C}

The saddle-point method provides a powerful analytical tool for estimating strong-field ionization amplitudes in the high-intensity regime. However, as demonstrated by Milošević et al.~\cite{Milosevic2006}, its applicability becomes questionable in the case of low-intensity fields and long-range binding potentials such as the Coulomb potential. The key limitation arises from the divergence of the matrix element $\mathbf{r} \cdot \mathbf{E}(t_s)$ at the saddle points $t_s$, signaling a breakdown of the conventional saddle-point approximation. 

In this appendix, we present a systematic derivation of the modified saddle-point approach, with particular emphasis on few-cycle laser pulses. Our treatment follows the formalism developed in~\cite{Milosevic2006}.

The matrix element for direct ionization in length gauge can be transformed through integration by parts. The derivation begins with the essential operator identity
\begin{equation}
\left[\bm{r} \cdot \bm{E}(t) + i \partial_t\right] \exp\left\{-i\left[\bm{p} + \bm{A}(t)\right] \cdot \bm{r}\right\} = 0
\end{equation}
This identity leads to the exact expression for the transition amplitude
\begin{equation}
	T_0(\mathbf{p}) = i \int_0^{T_p} \mathrm{d}t \, \tilde{\psi}_0(\bm{p} + \bm{A}(t)) \frac{\mathrm{d}S(t)}{\mathrm{d}t} e^{iS(t)} - \tilde{\psi}_0(\bm{p} + \bm{A}(0)) \left[e^{iS(T_p)} - e^{iS(0)}\right]
\end{equation}
where the classical action $S(t) \equiv S_{I_p, \bm{p}}(t)$ represents the time-integrated dynamical phase:
\begin{equation}
S(t) = \int_t^{T_p} \mathrm{d}t' \left[\frac{1}{2}(\bm{p} + \bm{A}(t'))^2 + I_p\right]
\end{equation}
The Fourier-transformed ground state wavefunction appears as:
\begin{equation}
\tilde{\psi}_0(\bm{k}) = (2\pi)^{-3/2} \int \mathrm{d}^3\bm{r} \, e^{i \bm{k} \cdot \bm{r}} \psi_0(\bm{r})
\end{equation}
The transition amplitude consists of two distinct physical contributions. The first term involves integration over the entire pulse duration, representing the volumetric ionization probability. The second term constitutes boundary contributions evaluated at the temporal endpoints of the laser interaction. For sufficiently long pulses where $\bm{A}(0) = \bm{A}(T_p) = 0$, these boundary terms become negligible in most physical scenarios.

The volumetric term contains the factor $\mathrm{d}S(t)/\mathrm{d}t$ which vanishes at saddle points $t_s$ defined by:
\begin{equation}
\left.\frac{\mathrm{d}S(t)}{\mathrm{d}t}\right|_{t=t_s} = 0
\end{equation}
This suggests potential singular behavior in the integrand. However, the Fourier transform $\tilde{\psi}_0(\bm{p} + \bm{A}(t))$ provides compensating factors that regularize the expression.


For a coulomb potential with binding energy $I_p = \kappa^2/2$, the ground-state wavefunction in momentum space take particular forms
\begin{equation}
\tilde{\psi}_0(\bm{p}) = \frac{2^{3/2} \kappa^{5/2}}{\pi(\kappa^2 + \bm{p}^2)^2} = \frac{\kappa^{5/2}}{\sqrt{2\pi} \left(\frac{\mathrm{d}S}{\mathrm{d}t}\right)^2}
\end{equation}
The different functional forms lead to distinct mathematical behavior when applying the saddle-point approximation. In the zero-range case, the $\mathrm{d}S/\mathrm{d}t$ dependence cancels exactly, while for Coulomb potentials a residual $(\mathrm{d}S/\mathrm{d}t)^{-1}$ factor remains. This residual singularity requires careful treatment through advanced contour integration techniques.

For Coulomb potentials, the amplitude expression becomes:
\begin{equation}
T_0(\mathbf{p}) = \frac{\sqrt{2}}{\pi} (2I_p)^{5/4} \left( \mathcal{I} - \left. \frac{\exp(iS)}{2\left(\frac{\mathrm{d}S}{\mathrm{d}t}\right)^2} \right|_{0}^{T_p} \right)
\end{equation}
where the principal integral is
\begin{equation}
\mathcal{I} = \frac{i}{2} \int_0^{T_p} \mathrm{d}t \, \left(\frac{\mathrm{d}S}{\mathrm{d}t}\right)^{-1} \exp(iS)
\end{equation}
The integration contour deforms into the complex plane to capture both endpoint contributions at $t = 0$ and $t = T_p$, along with saddle-point contributions at critical points $t_s$. The complete integral decomposes as:
\begin{equation}
\mathcal{I} \approx \mathcal{I}_{\text{endpoints}} + \sum_s \mathcal{I}_s
\end{equation}
The endpoint contributions through integration by parts become:
\begin{equation}
\mathcal{I}_{\text{endpoints}} = \left. \frac{\exp(iS)}{2\left(\frac{\mathrm{d}S}{\mathrm{d}t}\right)^2} \right|_0^{T_p} + \mathcal{O}\left(\frac{\mathrm{d}^2S}{\mathrm{d}t^2}\right)
\end{equation}
The dominant first term exactly cancels the boundary contribution in the amplitude expression.
Near each saddle point $t_s$, the action expands as
\begin{align}
S(t) &\approx S(t_s) + \frac{1}{2} S''(t_s) (t - t_s)^2 \\
\frac{\mathrm{d}S}{\mathrm{d}t} &\approx S''(t_s) (t - t_s)
\end{align}
The local integral transforms to:
\begin{equation}
\mathcal{I}_s = \frac{i}{2} [S''(t_s)]^{-1} e^{iS(t_s)} \int_{C_s} \frac{\mathrm{d}t}{t - t_s} \exp\left( \frac{i}{2} S''(t_s) (t - t_s)^2 \right)
\end{equation}
For $S''(t_s) = i |S''(t_s)| e^{i\alpha}$, the steepest descent path follows:
\begin{equation}
t - t_s = \rho e^{-i\alpha/2}, \quad \rho \in (-\infty, \infty)
\end{equation}
Application of the Sokhotski-Plemelj theorem yields the saddle-point contribution
\begin{equation}
\mathcal{I}_s = -\frac{\pi}{2} \frac{\exp(iS(t_s))}{S''(t_s)}
\end{equation}
Combining all contributions yields the physically meaningful result:
\begin{equation}
T_0(\mathbf{p}) = -2^{-1/2}(2I_p)^{5/4} \sum_s [S''(t_s)]^{-1} \exp(iS(t_s))
\end{equation}
This expression demonstrates that proper treatment of the apparent singularity leads to a well-defined result where each saddle-point contribution is weighted by the inverse curvature of the action.
