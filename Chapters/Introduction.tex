\chapter{Introduction}
\vspace*{0pt} % Ensures the content starts at the very top of the page

\begin{center}
	\textit{``Behind every atom of this world hides an infinite Universe.''} \\
	\vspace{0.5em} % Adds a small space between the quote and attribution
	\footnotesize \textbf{--- Rumi}
\end{center}

\vspace{2em} % Adds some space after the quote
\section{Background and Context}
Since the dawn of civilization, humanity has been driven by an intrinsic desire to comprehend the fundamental laws that govern the natural world. These principles not only dictate the behavior of natural phenomena but also shape the very fabric of our existence. By understanding these laws, we gain the ability to manipulate and harness natural processes, leading to innovations that simplify our lives and revolutionize the way we live. Central to many of these advancements is our ability to precisely control and engineer matter, enabling the development of transformative technologies that have reshaped society.

One of the most profound and effective methods for probing and manipulating matter at its most fundamental level is through the interaction of light and matter \cite{Franken1961Aug,Chu1985Jul,Ebbesen1998,Goulielmakis2008,Assion1998}. Among the various techniques available, \textit{photoionization} stands out as a direct and powerful approach for investigating the electronic structure of atoms, molecules, and solid materials. This phenomenon, first explained by Albert Einstein in 1905 as part of his pioneering work on the photoelectric effect, describes the process by which matter absorbs photons—particles of light—and subsequently releases electrons \cite{Einstein1905}. This interaction results in the transformation of the matter into a charged, or ionic, state.

For photoionization to occur, the energy of the absorbed photon must exceed the binding energy that holds the electron within its atomic or molecular structure. When this condition is met, the electron is ejected, becoming what is known as a \textit{photoelectron}. By analyzing the properties of these photoelectrons—such as their kinetic energy and angular distribution—we can extract critical insights into the electronic configuration and behavior of the original material. This information is invaluable for understanding the fundamental properties of matter.

\section{Historical Development of Strong Field Ionization}
The dynamics of interactions between atoms and photons are profoundly influenced by the energy of the incident photons in the applied radiation field. These interactions exhibit distinct behaviors depending on whether the photon energy \(\hbar \omega\) exceeds or falls below the ionization potential \(I_p\) of the bound electron in an atom. Here \(\hbar\) and \(\omega\) are the Planck's constant and frequency of the photon respectively. When \(\hbar \omega > I_p\), the process is dominated by \textit{single-photon ionization}. In this regime, the kinetic energy \(E_k\) of the ejected electron is given by the relation
\[
E_k = \hbar \omega - I_p.
\]
 This equation encapsulates the fundamental principle that the excess energy of the photon, beyond what is required to overcome the binding energy of the electron, is transferred to the electron as kinetic energy.

In contrast, when the photon energy \(\hbar \omega\) is less than the ionization potential \(I_p\), single-photon ionization is no longer possible. Instead, ionization can occur through the simultaneous absorption of multiple photons, a process known as \textit{multi-photon ionization}. This concept was first theorized by Paul Dirac in 1927, who proposed that electrons could absorb multiple photons collectively to achieve the energy necessary for ionization \cite{Dirac1927}. Initially considered a theoretical curiosity, multi-photon ionization was later confirmed as a genuine physical phenomenon through experimental observations \cite{agostini1988photoelectric}.

The probability of an atom absorbing \(n\) photons from a radiation field of intensity \(I\), for each absorbed photon, is governed by the expression $P_n \propto \left(\frac{I}{I_{\text{a.u.}}}\right)^n$, where \(I_{\text{a.u.}}\) is a characteristic value representing the intensity corresponding to the electric field that binds an electron in a Hydrogen atom. The minimum number of photons \(n_{\text{th}}\) required for ionization is determined by the threshold condition $n_{\text{th}} = \left\lfloor \frac{I_p}{\hbar \omega} \right\rfloor + 1,$ where \(\lfloor \cdot \rfloor\) represents the floor function. This equation highlights the nonlinear dependence of the ionization process on the intensity of the radiation field.

The advent of the Ruby laser in 1960, pioneered by Theodore Maiman \cite{Maiman1960}, marked a turning point in the study of multi-photon processes \cite{Voronov1965}. The availability of intense, coherent laser sources enabled researchers to explore higher-order multi-photon ionization phenomena in greater detail. A landmark achievement in this field was the discovery of \textit{above-threshold ionization} (ATI) by Agostini and colleagues in 1979 \cite{Agostini1979}. ATI occurs when an atom absorbs more photons than the minimum required for ionization, resulting in the emission of electrons with kinetic energies significantly higher than those predicted by simple perturbation theory \cite{petite1987intensity}. The observation of ATI challenged the conventional understanding of multi-photon ionization, as it revealed the limitations of perturbative approaches in describing highly nonlinear processes \cite{Keldysh1964}. While perturbation theory provides a useful framework for understanding low-order multi-photon interactions \cite{Bebb1966}, it fails to account for the complex dynamics observed in experiments involving intense laser fields \cite{Corkum1993,Krausz2009}. Consequently, non-perturbative methods, such as the numerical solution of the time-dependent Schrödinger equation, have become essential for accurately modeling these phenomena \cite{torlina2012,javanainen1988,jheng2022,bauer2006}.

The evolution of our understanding of atom-photon interactions underscores the intricate relationship between theory and experiment in advancing scientific knowledge. While early theoretical frameworks provided foundational insights, experimental breakthroughs often revealed new complexities that necessitated the development of more sophisticated models. The study of multi-photon ionization, in particular, exemplifies the interplay between nonlinear optical processes and quantum mechanical principles, offering a rich area of research with implications for fields ranging from atomic physics to quantum chemistry \cite{Pabst2013,Fang2023,Wang2024,Itatani2004,Lein2002,Smirnova2009}.

\section{Overview of Strong Field Ionization}
Strong-field ionization (SFI) is a fundamental process in atomic, molecular, and optical physics that occurs when an atom or molecule is exposed to an intense laser field, typically with electric field strengths comparable to the atomic unit (\(50 \, \text{V}/\mathring{\text{A}}\)). In this regime, the electric field of the laser becomes strong enough to significantly distort the Coulomb potential binding the electron to the nucleus, leading to ionization through mechanisms such as tunneling or over-the-barrier escape. This phenomenon has profound implications for understanding light-matter interactions and has become a cornerstone of modern physics, enabling advancements in fields such as attosecond science, high-harmonic generation, and ultrafast spectroscopy.

The study of strong-field ionization is not only of theoretical interest but also has practical applications in areas such as laser technology, plasma physics, and quantum chemistry. By manipulating the intensity, wavelength, and duration of laser pulses, the ionization process can be control to explore the dynamics of electrons in atoms and molecules with unprecedented precision. This section provides an in-depth discussion of the fundamental principles underlying strong-field ionization and its diverse applications in modern physics.

\subsection{Fundamental Principles}
The behavior of electrons in strong laser fields is governed by the interplay between the laser's electric field and the atomic or molecular potential. A key theoretical framework for understanding strong-field ionization is provided by the \textit{Keldysh parameter} \(\gamma\), a dimensionless quantity defined as \cite{Keldysh1964}
\[
\gamma = \sqrt{\frac{I_p}{2U_p}},
\]
where \(I_p\) is the ionization potential of the atom or molecule, and \(U_p\) is the ponderomotive energy, which represents the cycled-average kinetic energy of an electron oscillating in the laser field. The Keldysh parameter distinguishes between three primary ionization regimes (see Fig. \ref{fig:MPI}). 
\begin{figure}
	\centering
	\includegraphics[width=0.9\textwidth]{gfx/Final/Introduction/MPI.pdf}
	\caption{Schematic illustration of strong-field ionization mechanisms driven by a two-cycle circularly polarized  $800~\mathrm{nm}$ laser pulse. The blue lines depict the total potential, combining the Coulomb potential and the laser-induced potential at the peak electric field. The red dotted lines represent the ionization threshold, including the ponderomotive potential. The hydrogen $1s$ radial electron probability distribution (green), with its energy level aligned to the lower part of the potential well. Red arrows indicate the multi-photon ionization (MPI) process, illustrating the required photon absorption (ph) for ionization at varying intensities. 		
		(a) At an intensity of \( 8 \times 10^{11} \) W/cm\(^2\) (\(\gamma = 27.85\)), ionization is dominated by MPI, as the potential barrier remains largely intact, suppressing tunneling. (b) For \( 8 \times 10^{13} \) W/cm\(^2\) (\(\gamma = 2.78\)), the potential barrier is significantly distorted, enabling tunneling ionization (black arrow), while additional photon absorption is required to surpass the ionization threshold. (c) At \( 8 \times 10^{14} \) W/cm\(^2\) (\(\gamma = 0.88\)), tunneling dominates, and MPI becomes negligible. The barrier is further suppressed, leading to over-the-barrier ionization (OBI) at higher intensities, where the electron escapes without tunneling, primarily near the peak laser intensity.}
	\label{fig:MPI}
\end{figure}

When \(\gamma \gg 1\) (Fig.\ref{fig:MPI}a), the ionization process is dominated by the simultaneous absorption of multiple photons. This regime is characterized by a low-intensity laser field, where the photon energy \(\hbar \omega\) is much smaller than the ionization potential \(I_p\). Multiphoton ionization was first observed experimentally in the 1960s and was theoretically explained by Keldysh, who developed a formalism to describe the transition from multiphoton to tunneling ionization \cite{Agostini1968}. MPI has been extensively studied in a variety of systems, including small molecules, clusters, and biomolecules, providing valuable insights into their electronic structures and dynamics \cite{Mainfray_1991,Sansone2006,Grotemeyer1989,Echt1985}.

For \(\gamma \approx 1\) (Fig.\ref{fig:MPI}b), the electric field is strong enough to distort the atomic potential barrier, enabling the electron to tunnel through it. This phenomenon was first described by Fowler and Nordheim in the context of electron emission from metals \cite{Fowler1928} and was later extended to atoms and molecules through the development of the Ammosov-Delone-Krainov (ADK) theory \cite{Oppenheimer1928,Ammosov1986}. The ADK theory provides a semi-classical framework for calculating ionization rates in non-hydrogenic atoms, taking into account the effects of multiple electrons and atomic centers. Further refinements, such as the Keldysh-Faisal-Reiss (KFR) theory, incorporate additional factors like orbital geometry and atomic interactions, offering a more comprehensive understanding of ionization rates \cite{Keldysh1964,Faisal_1973,Reiss1980}.
	
When the Keldysh parameter satisfies $\gamma \ll 1$ (Fig.~\ref{fig:MPI}c), the ionization process occurs predominantly via quantum tunneling through the laser-distorted potential barrier. In this regime, the laser field varies slowly compared to the electron’s tunneling time, allowing the electron to escape even though the barrier remains finite. For a fixed intensity, this tunneling limit can be reached by decreasing the laser frequency (i.e., increasing the wavelength), without necessarily entering the over-the-barrier ionization (OBI) regime. Only when the electric field becomes sufficiently strong to suppress the potential barrier below the binding energy does OBI occur. Such extreme field conditions, often realized with few-femtosecond laser pulses, enable the study of ultrafast electron dynamics and non-thermal distributions, providing a deeper understanding of the underlying physical processes~\cite{Ivanov2016,Cohen2001}.


\subsection{Applications in Modern Physics}
The study of strong-field ionization extends far beyond its theoretical foundations, offering a wealth of practical applications that have revolutionized our understanding of atomic and molecular systems. By subjecting matter to intense laser fields, researchers can explore regimes where traditional perturbative approaches fail, revealing new physical phenomena that are both complex and enlightening. Among these, ATI stands out as a cornerstone of strong-field physics, providing a unique lens through which to examine the nonlinear interaction between light and matter.

ATI occurs when an electron absorbs more photons than the minimum required for ionization, resulting in kinetic energies that exceed those predicted by conventional models. This phenomenon challenges the simplistic view of ionization as a single-photon process and instead highlights the intricate dynamics of multi-photon absorption in strong laser fields. The study of ATI has not only deepened our understanding of quantum mechanics but has also paved the way for advancements in ultrafast spectroscopy and attosecond science. By analyzing the energy and angular distributions of photoelectrons, researchers can extract detailed information about the electronic structure and dynamics of atoms and molecules.

\begin{figure}
	\centering
	\includegraphics[width=1.\textwidth]{gfx/Final/Introduction/threestep.pdf}
	\caption{Three-step model of laser-induced ionization dynamics in krypton, modeled using a Coulombic binding potential $V(r) = -1/r$. The atom is exposed to an intense laser field of wavelength $800~\mathrm{nm}$ and peak intensity $10^{14}~\mathrm{W/cm^2}$. (Left) \textbf{Tunnel Ionization:} An electron (red dot) escapes the atomic potential well by tunneling through the laser-distorted potential barrier (red arrow). (Mid) \textbf{Propagation in the Continuum:} The ionized electron moves in the laser field, tracing a trajectory away from the nucleus. (Right) \textbf{Re-collision:} The electron may return to the nuclear vicinity, leading to processes such as High-Harmonic Generation (HHG), high-order ATI, or double ionization (multiple red dots), illustrating the intricate post-ionization dynamics.}

	\label{fig:threestep}
\end{figure}

A particularly intriguing aspect of strong-field ionization is the role of recollision phenomena, where ionized electrons are driven back to their parent ions by the oscillating laser field. These recollisions give rise to a variety of effects that are well explained by the three-step model , proposed by Corkum in 1993, as show in Fig. \ref{fig:threestep}. For example,  laser-induced electron diffraction \cite{Spanner_2004,Yurchenko2004} exploits the wave-like nature of recolliding electrons to create interference patterns that encode information about the spatial arrangement of atoms within a molecule. Similarly, electron holography \cite{Huismans2011} uses these interference effects to reconstruct detailed images of molecular structures, offering insights into processes such as chemical bonding and reaction dynamics.

Recollision also plays a central role in high-order above-threshold ionization (HATI), a process in which rescattered electrons gain additional energy through interactions with the parent ion. This results in photoelectrons with kinetic energies far beyond those predicted by simple tunneling models \cite{Paulus1994}. HATI provides a powerful platform for studying the non-perturbative nature of strong-field interactions, shedding light on the complex interplay between the laser field and the ionized electron. Additionally, recollision-driven processes such as non-sequential multiple ionization\cite{Walker1994,Liu2021,Liu2022} have revealed the correlated behavior of electrons in strong fields, where a single recolliding electron can liberate multiple electrons from the same ion. These findings have profound implications for understanding electron-electron interactions in complex systems.

While the primary focus of this thesis is on ATI phenomena, it is worth noting the broader impact of strong-field ionization on other areas of research. One such area is high-harmonic generation (HHG), a process in which recolliding electrons generate coherent radiation at integer multiples of the laser frequency. Although HHG is not the central theme of this work, it exemplifies the rich and diverse phenomena that arise from strong-field interactions. HHG has enabled the development of attosecond light sources, which are used to probe ultrafast electron dynamics in atoms, molecules, and solids \cite{Weber2021,Minneker2021,Paufler2019}. The ability to control and observe electron dynamics on attosecond timescales has opened new frontiers in modern physics \cite{Birger2017}. By leveraging the unique capabilities of strong-field ionization, researchers can probe ultrafast processes such as electron tunneling, wave packet dynamics, and molecular fragmentation with unprecedented precision. These advancements have far-reaching implications for fields such as quantum chemistry, materials science, and ultrafast optics, where understanding and manipulating electron behavior is essential for developing new technologies.



%%%%%%%%%%%%%%%%%%%%%%%%%%%%%%% New Section %%%%%%%%%%%%%%%%%%%%%%%%%%%%%%


\section{Above-Threshold Ionization: Concepts and Significance}

ATI is a fundamental phenomenon in strong-field physics that occurs when an atom or molecule absorbs more photons than the minimum number required for ionization. This process, first observed in the late 1970s, has since become a cornerstone in the study of light-matter interactions, particularly in the regime of intense laser fields. ATI provides critical insights into the dynamics of electrons under extreme electromagnetic conditions and has significant implications for understanding nonlinear optical processes, electron correlation, and quantum mechanics in intense fields.

\subsection{Experimental Observations and Implications}

The experimental discovery of ATI marked a turning point in the study of laser-atom interactions. Early experiments using intense laser pulses revealed that ionized electrons could be ejected with kinetic energies corresponding to the absorption of additional photons beyond the ionization threshold (see Fig. \ref{fig:ati}) \cite{Agostini1979,Kruit1983}. This observation contradicted the classical picture of photoionization, where electrons were expected to absorb only the minimum energy required to escape the atomic potential. Instead, the energy spectrum of emitted electrons exhibited discrete peaks, each separated by the photon energy of the incident laser field \cite{Korneev2012}. These peaks, known as ATI peaks, provided direct evidence of multiphoton processes in the strong-field regime \cite{BECKER200235}.
\begin{figure}
	\centering
	\includegraphics[width=0.7\textwidth]{gfx/Final/Introduction/ati.pdf}
	\caption{Potential energy curve and above-threshold ionization (ATI) spectrum of an Krypton atom in a strong laser field. The blue curve represents the atomic potential energy, while small red arrows indicate the absorption of photons enabling the electron to surpass the ionization potential ($I_p$). The inset displays the ATI spectrum, with peaks corresponding to the absorption of excess photons beyond $I_p$, measured in units of $\hbar\omega$.}
	\label{fig:ati}
\end{figure}

The implications of ATI are profound. First, it demonstrated the breakdown of perturbation theory in intense laser fields, necessitating the development of non-perturbative theoretical frameworks. Second, ATI spectra serve as a sensitive probe of the laser field parameters, such as intensity, wavelength, and pulse duration, as well as the atomic or molecular structure \cite{Noslen2015,Petite1987,Milosevic2007,Corkum1989,Marchenko_2010,Haiying2023}. For instance, the angular distribution of ATI electrons encodes information about the symmetry of the initial electronic state and the influence of the laser field on the ionization dynamics \cite{Birger2023}. Furthermore, ATI has been instrumental in the development of attosecond science, as the high-energy electrons generated through ATI can be used to probe ultrafast processes in atoms, molecules, and solids.

Recent advances in experimental techniques, such as velocity map imaging and coincidence spectroscopy, have enabled detailed measurements of ATI spectra with unprecedented resolution \cite{Trabert2023,Kang2022}. These experimental techniques have revealed subtle features, such as low-energy structures and rescattering plateaus, which have deepened our understanding of strong-field ionization mechanisms \cite{Paulus1994,Quan2009}. Moreover, ATI has found practical applications in the generation of high-harmonic radiation and the development of compact electron accelerators, highlighting its significance beyond fundamental research.

\subsection{Theoretical Approaches of ATI}

The theoretical description of ATI is rooted in quantum mechanics and the interaction of atoms with intense electromagnetic fields. Two primary models have been developed to explain ATI: the \textit{strong-field approximation} (SFA) and the \textit{semiclassical recollision model}. These approaches provide complementary insights into the underlying physics of ATI, enabling researchers to interpret experimental observations and refine theoretical frameworks.

The strong-field approximation, also known as the Keldysh-Faisal-Reiss theory, treats the laser field as a classical electromagnetic wave and describes the ionization process using time-dependent perturbation theory. In this model, the electron is assumed to be instantaneously liberated from the atomic potential via tunneling ionization, after which it propagates as a free particle in the laser field. The SFA successfully predicts the existence of ATI peaks and their dependence on laser parameters, such as intensity and wavelength \cite{Hasovic2012,Milosevic2019,Austin2012}. However, it neglects the influence of the atomic potential on the ejected electron, which can lead to discrepancies with experimental observations, particularly in the low-energy regime \cite{Arbo2009,Bauer2006TF}.

The semiclassical recollision model as shown in Fig. \ref{fig:threestep}, often associated with high-harmonic generation (HHG), also provides valuable insights into ATI. While the three-step model is more commonly used to describe HHG, it can be adapted to explain certain aspects of ATI \cite{ShvetsovShilovski2016,Min2014}, particularly the rescattering mechanism \cite{Rook2024}. In this model, the ionization process is divided into three steps: (1) \textit{tunneling ionization}, where the electron escapes the atomic potential through the barrier suppressed by the laser field; (2) \textit{acceleration in the laser field}, where the electron gains energy from the oscillating electric field; and (3) \textit{recollision}, where the electron may return to the parent ion. In the context of ATI, recollision can lead to elastic or inelastic scattering, contributing to the high-energy plateau observed in ATI spectra. However, the primary focus of the three-step model remains on HHG and other higher-order processes.

Both the SFA and recollision models have been extended and refined to incorporate additional effects, such as Coulomb focusing, quantum interference, and multielectron dynamics \cite{Yang2020,Klaiber2017,Werby2022,Boning2020,boning2023}. For instance, the inclusion of the Coulomb potential in the SFA has significantly improved its agreement with experimental data, particularly for low-energy electrons. Furthermore, advanced numerical methods, such as \textit{time-dependent density functional theory} (TDDFT) and solutions to the \textit{time-dependent Schrödinger equation} (TDSE), have enabled precise simulations of ATI in complex systems. The SFA is employed in this thesis due to its computational efficiency and ability to provide physical insight into strong-field ionization dynamics while maintaining reasonable agreement with experimental observations.





%%%%%%%%%%%%%%%%%%%%%%%%%%%%%%% New Section %%%%%%%%%%%%%%%%%%%%%%%%%%%%%%


\section*{Motivation for the Study}

The study of strong-field ionization and ATI lies at the heart of modern atomic, molecular, and optical physics. As laser technology continues to advance, enabling the generation of ultrashort and high-intensity pulses, the interaction of light with matter has entered a regime where traditional perturbative approaches no longer suffice. Understanding the dynamics of electrons in such intense fields is not only a fundamental scientific challenge but also a gateway to groundbreaking applications in ultrafast science, attosecond physics, and laser-driven particle acceleration.

One of the primary motivations for this research is the need to unravel the complex mechanisms underlying strong-field ionization. The ATI process provides a unique window into the quantum dynamics of electrons in extreme electromagnetic environments. By studying ATI, we gain insights into the interplay between quantum mechanics and classical electrodynamics, as well as the role of electron correlation and nondipole effects in ionization processes.

From a practical perspective, the implications of strong-field ionization extend far beyond fundamental physics. ATI plays a crucial role in the generation of high-harmonic radiation, which is the foundation of attosecond science. Attosecond pulses, in turn, enable the real-time observation of electron dynamics in atoms, molecules, and solids, opening new frontiers in ultrafast spectroscopy and imaging. Furthermore, the high-energy electrons produced through ATI have potential applications in compact electron accelerators and advanced radiation sources, which could revolutionize fields such as materials science, chemistry, and medicine.

Another key motivation for this study is the exploration of structured light, particularly twisted light beams, in strong-field processes. Unlike conventional plane waves, twisted light carries orbital angular momentum, offering new degrees of freedom for controlling ionization dynamics. Understanding how such structured light interacts with matter could lead to novel techniques for manipulating electron motion and designing tailored laser-matter interactions. This has implications for quantum information processing, precision metrology, and the development of next-generation optical technologies.

Finally, this research is driven by the desire to bridge the gap between theory and experiment. While significant progress has been made in understanding strong-field ionization, many questions remain unanswered. For instance, the role of nondipole effects, the influence of laser pulse parameters, and the behavior of electrons in complex systems are areas that require further exploration. Nondipole effects emerge from the spatial variation of the laser field, where the magnetic component exerts a Lorentz force on the ionized electron. By developing advanced theoretical models and numerical techniques, this thesis aims to provide a deeper understanding of these phenomena and offer predictions that can guide future experimental investigations.

In summary, the motivation for this study is multifaceted, encompassing both fundamental scientific inquiry and practical applications. By advancing our understanding of strong-field ionization and ATI, this research contributes to the broader goals of ultrafast and high-intensity laser physics, paving the way for understanding the ionization in greater details.


%%%%%%%%%%%%%%%%%%%%%%%%%%%%%%% New Section %%%%%%%%%%%%%%%%%%%%%%%%%%%%%%


\section{Structure of the Thesis}

This thesis is systematically organized into several chapters, each dedicated to exploring a critical aspect of strong-field ionization. The progression of the chapters is designed to guide the reader from foundational concepts to advanced theoretical frameworks, numerical methodologies, and discussions on experimental relevance and future prospects. Below is an overview of the structure and content of each chapter.

In Chapters 2 and 3, we establish the theoretical framework for strong-field ionization. Chapter 2 begins with an overview of electromagnetic waves, covering Maxwell’s equations, plane-wave solutions, and the distinctive features of twisted light beams in strong-field processes. Chapter 3 then introduces the strong-field approximation as a key theoretical approach, elaborating on dipole and nondipole effects, as well as the role of the Lorentz force in ionization dynamics. Additionally, the chapter examines different gauge formulations, laying a rigorous foundation for the analysis of laser-atom interactions.

Chapter 4 focuses on the mathematical and computational techniques employed in this research. The Jacobi-Anger expansion is introduced as a powerful method for analyzing twisted wavefields. The saddle point approximation (SPA) is discussed in detail, including its derivation, validity, and limitations. The chapter also covers the numerical implementation of saddle point equations, utilizing root-finding algorithms and iterative methods. Additionally, computational techniques for calculating photoelectron momentum distributions are presented, providing a bridge between theory and numerical results.

Chapter 5 presents the core findings of this research, offering a detailed exploration of various aspects of ATI. Topics include ionization dynamics in few-cycle laser pulses, the role of ellipticity in saddle point solutions, and nonlinear interference effects in ATI peaks and momentum distributions. Nondipole effects and their dependence on laser pulse parameters are analyzed, alongside the interaction of twisted light with atoms and its impact on ATI peak positions. The chapter also examines asymmetries in photoelectron momentum distributions, providing insights into their dependence on laser characteristics. These results are discussed in the context of both theoretical predictions and experimental observations.

The final chapter summarizes the key contributions of this thesis to the field of strong-field ionization. It highlights the implications of the findings for ultrafast and high-intensity laser physics. The chapter concludes with a forward-looking perspective, proposing future research directions. 




