%*******************************************************
% Abstract
%*******************************************************
%\renewcommand{\abstractname}{Abstract}
\pdfbookmark[1]{Abstract}{Abstract}
% \addcontentsline{toc}{chapter}{\tocEntry{Abstract}}
\begingroup
\let\clearpage\relax
\let\cleardoublepage\relax
\let\cleardoublepage\relax

\chapter*{Abstract}
Strong-field ionization (SFI) serves as a cornerstone in understanding ultrafast electron dynamics and light-matter interactions under the influence of intense laser fields. This thesis presents a rigorous theoretical investigation of SFI, focusing on above-threshold ionization (ATI)—a process in which ionized electrons absorb additional photons above the ionization threshold, leading to discrete high-energy peaks in the photoelectron spectrum, as well as quantum interference phenomena and nondipole effects across different laser field configurations.

The study first examines SFI driven by few-cycle laser pulses within the strong-field approximation (SFA), systematically comparing dipole and nondipole regimes. While the dipole approximation effectively captures quantum interference structures in photoelectron momentum distributions (PMDs), nondipole effects induce significant momentum shifts and asymmetries, particularly along the laser propagation axis. Theoretical predictions for so-called peak shifts, specially for a few-cycle pulse, show enhanced experimental agreement upon incorporating nondipole corrections.

To unravel these strong-field dynamics, this work employs two complementary analytical approaches: (1) The Jacobi-Anger expansion provides a complete decomposition of the ionization transition amplitude into photon orders in terms of Bessel functions. (2) The saddle-point method isolates dominant quantum orbits (stationary solutions to the classical action) that contribute maximally to the ionization amplitude, offering both physical insight and improved computational efficiency in resolving PMDs within different laser configurations. The validity regimes, advantages, and limitations of these methods are critically assessed. A detailed analysis of nonlinear interference effects reveals how electron dynamics are governed by the interplay of fundamental frequencies concealed within a pulse. The influence of the carrier-envelope phase on interference structures is quantified, while an increasing number of optical cycles is shown to constrain PMD features and reshape the energy-resolved ionization spectrum.

In the end, the theoretical framework is generalized to structured light fields, particularly twisted Bessel pulses carrying orbital angular momentum (OAM). Within the SFA and saddle-point formalism, analytical expressions for the ionization amplitude are derived for hydrogenic targets. The impact of the Bessel pulse's opening angle and OAM on shaping PMDs and ATI peak positions is systematically investigated, highlighting their role as tunable parameters for electron emission.
 

\vfill
\newpage\null\thispagestyle{empty}\newpage

\begin{otherlanguage}{ngerman}
\pdfbookmark[1]{Zusammenfassung}{Zusammenfassung}
\chapter*{Zusammenfassung}

Die Starkfeldionisation (engl. SFI) bildet einen Grundpfeiler zum Verständnis ultraschneller Elektronendynamik und Licht-Materie-Wechselwirkungen unter dem Einfluss intensiver Laserfelder. Diese Dissertation präsentiert eine rigorose theoretische Untersuchung der SFI mit Fokus auf die Above-Threshold-Ionisation (engl. ATI) – einen Prozess, bei dem ionisierte Elektronen zusätzliche Photonen oberhalb der Ionisationsschwelle absorbieren, was zu diskreten, hochenergetischen Peaks im Photoelektronenspektrum führt – sowie auf Quanteninterferenzphänomene und Nicht-Dipol-Effekte in verschiedenen Laserfeldkonfigurationen.

Die Studie untersucht zunächst die SFI, angetrieben durch Laserpulse mit wenigen optischen Zyklen innerhalb der Starkfeldnäherung (engl. SFA), und vergleicht systematisch Dipol- und Nicht-Dipol-Regime. Während die Dipolnäherung Quanteninterferenzstrukturen in Photoelektronenimpulsverteilungen (engl. PMDs) effektiv beschreibt, verursachen Nicht-Dipol-Effekte signifikante Impulsverschiebungen und Asymmetrien, insbesondere entlang der Ausbreitungsachse des Lasers. Theoretische Vorhersagen für sogenannte Peak-Shifts, insbesondere für Pulse mit wenigen optischen Zyklen, zeigen eine verbesserte Übereinstimmung mit Experimenten bei Einbeziehung von Nicht-Dipol-Korrekturen.

Um diese Starkfelddynamiken zu entschlüsseln, werden in dieser Arbeit zwei komplementäre analytische Ansätze verwendet: (1) Die Jacobi-Anger-Entwicklung liefert eine vollständige Zerlegung der Ionisationsübergangsamplitude in Photonenordnungen mittels Besselfunktionen. (2) Die Sattelpunktsmethode isoliert dominante Quantenbahnen (stationäre Lösungen der klassischen Wirkung), die maximal zur Ionisationsamplitude beitragen, und bietet sowohl physikalische Einsichten als auch verbesserte Recheneffizienz bei der Auflösung von PMDs in verschiedenen Laserkonfigurationen. Die Gültigkeitsbereiche, Vorteile und Grenzen dieser Methoden werden kritisch bewertet. Eine detaillierte Analyse nichtlinearer Interferenzeffekte zeigt, wie die Elektronendynamik durch das Zusammenspiel fundamentaler Frequenzen innerhalb eines Pulses bestimmt wird. Der Einfluss der Träger-Einhüllenden-Phase auf Interferenzstrukturen wird quantifiziert, während eine zunehmende Anzahl optischer Zyklen die Merkmale der PMDs einschränkt und das energieaufgelöste Ionisationsspektrum verändert.

Abschließend wird der theoretische Rahmen auf strukturierte Lichtfelder verallgemeinert, insbesondere auf getwistete Bessel-Pulse mit orbitalem Drehimpuls (engl. OAM). Innerhalb der SFA und des Sattelpunktsformalismus werden analytische Ausdrücke für die Ionisationsamplitude für wasserstoffähnliche Zielatome hergeleitet. Der Einfluss des Öffnungswinkels und des OAM der Bessel-Pulse auf die Gestalt von PMDs und ATI-Peak-Positionen wird systematisch untersucht, wobei ihre Rolle als einstellbare Parameter für die Elektronenemission hervorgehoben wird.




\end{otherlanguage}

\endgroup

\vfill
